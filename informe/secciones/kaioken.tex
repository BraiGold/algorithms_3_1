\section{KaioKen}

	\subsection{Descripción del Problema}
		Sea $n \in \mathds{N}$, $c: conj(lista(nat))$. El objetivo de este problema es encontrar la menor cantidad de listas necesarias que cumplan con las siguientes condiciones: 
		    \begin{itemize}
                \item ($ \forall l \in c$) tam($l$) = $n$
                \item ($ \forall l \in c$) ($ \forall i < n$) $l[i]$ = 1 $\vee$ $l[i]$ = 2
                \item ($ \forall i < n$) ($ \forall j < n$) $\exists$ ($l \in c$) $l[i] \neq l[j]$
            \end{itemize}

	
    \subsection{Desarrollo}


    \subsection{Complejidad}

    \begin{algoritmo}{nombreAlgoritmo}{parametros}{tipoSalida}
  % Crear e inicializar variables:
  \tipo{tipoVariable} $nombreVariable \gets valor$ \;
  % Nota: las líneas sin comentario deben terminar en \;

  % Comentarios:
  Contenido de la linea \tcp*[h]{Comentario pegado al texto, onda C++} \com*{Comentario contra el margen}
  Contenido de la linea \com*[h]{Comentario pegado al texto} \tcp*{Comentario contra el margen, onda C++}

  % If/then:
  \If{guarda}{
    Codigo \;
  }

  % If/then/else:
  \eIf(\com*[f]{Comentario opcional}){guarda}{
    Codigo (then) \;
  }(\com*[f]{Otro comentario opcional}){
    Codigo (else) \;
  }

  % If/then/elseif:
  \uIf{guarda}{
    Codigo (then) \;
  }\ElseIf{otra guarda}{
    Codigo (else) \;
  }

  % While:
  \While(\com*[f]{Comentario opcional}){guarda}{
    Codigo \;
  }

\end{algoritmo}


    \subsection{Código}


    \subsection{Experimentación}
		El objetivo de este experimento fue extraer conclusiones acerca de la variación en el tiempo de cómputo requerido por el algoritmo para distintos valores de $n$, con el fin de determinar su complejidad. 
		Para realizar las mediciones se  utilizaron las funciones provistas a tal efecto por la cátedra. Con el fin de evitar posibles errores en las mismas, cada una se repitió 10 veces, considerando luego el promedio entre los valores obtenidos.
		En este caso, se consideraron los siguientes valores de $n$: 


		\subsubsection*{Conclusión}
			Se puede observar en el gráfico que la curva $n log n$ está por encima de la curva que forman las ejecuciones del programa. Por lo tanto, se demuestra empiricamente que la complejidad del programa es O(nlogn)
